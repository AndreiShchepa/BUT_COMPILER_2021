% Author: Andrej Binovsky <xbinov00@stud.fit.vutbr.cz>
% Author: Zdenek Lapes <xlapes02@stud.fit.vutbr.cz>
% Author: Andrei Shchapaniak <xshcha00@stud.fit.vutbr.cz>
% Author: Richard Gajdosik <xgajdo33@stud.fit.vutbr.cz>


\documentclass[a4paper, 11pt]{article}


\usepackage[slovak]{babel}
\usepackage[utf8]{inputenc}
\usepackage[left=2cm, top=3cm, text={17cm, 24cm}]{geometry}
\usepackage{times}
\usepackage{verbatim}
\usepackage{enumitem}
\usepackage{graphicx} % vkladanie obrazkov
\usepackage[unicode]{hyperref}
\hypersetup{
    colorlinks = false,
    hypertexnames = false,
}

\newcommand{\RNum}[1]{\uppercase\expandafter{\romannumeral #1\relax}} % makro na sázení římských čísel

%%%%%%%%%%%%%%%%%%%%%%%%%%%%%%%% Andrei import -- LL tabulka/pravidla %%%%%%%%%%%%%%%%%%%%%%%%%%%%%%%%

\usepackage{xcolor}
\usepackage{float}
\usepackage{listings}
\usepackage{amsmath}
\usepackage{bm}
\usepackage{makecell}
\usepackage{cprotect}
\usepackage{multirow}
\usepackage{array}
\usepackage{changepage}
\usepackage{color, colortbl}
\definecolor{LightCyan}{rgb}{0.88,1,1}
\definecolor{White}{rgb}{255,255,255}

\def\nonterm #1{\boldmath{$<$}\textbf{#1}\boldmath{$>$}\space}
\def\rot #1{{\rotatebox[origin=c]{90}{\term{#1}}}}
\def\term #1{\texttt{#1}\space}
\newcommand{\arrow} {$\color{red} \rightarrow$\space}
\newcommand{\unsc} {\underline{\hspace{0.2cm}}}
\newcolumntype{P}[1]{>{\centering\arraybackslash}p{#1}}


%%%%%%%%%%%%%%%%%%%%%%%%%%%%%%%% Andrei import -- LL tabulka/pravidla %%%%%%%%%%%%%%%%%%%%%%%%%%%%%%%%

\begin{document}


    %%%%%%%%%%%%%%%%%%%%%%%%%%%%%%%% Titulná stránka %%%%%%%%%%%%%%%%%%%%%%%%%%%%%%%%
    \begin{titlepage}
        \begin{center}
            \includegraphics[width=0.77\linewidth]{src/FIT_logo.pdf} \\

            \vspace{\stretch{0.382}}

            \Huge{Projektová dokumentácia} \\
            \LARGE{\textbf{Implementácia prekladača imperatívneho jazyka IFJ21}} \\
            \Large{Tým 082, varianta \RNum{2}}
            \vspace{\stretch{0.618}}
        \end{center}

        \begin{minipage}{0.4 \textwidth}
        {\Large \today}
        \end{minipage}
        \hfill
        \begin{minipage}[r]{0.6 \textwidth}
            \Large
            \begin{tabular}{l l l}
                \textbf{Andrei Shchapaniak} & \textbf{(xshcha00)} & \quad 25\,\% \\
                Andrej Binovsky & (xbinov00) & \quad 25\,\% \\
                Zdenek Lapes & (xlapes02) & \quad 25\,\% \\
                Richard Gajdosik & (xgajdo33) & \quad 25\,\% \\
            \end{tabular}
        \end{minipage}
    \end{titlepage}



    %%%%%%%%%%%%%%%%%%%%%%%%%%%%%%%% Obsah %%%%%%%%%%%%%%%%%%%%%%%%%%%%%%%%
    \pagenumbering{roman}
    \setcounter{page}{1}
    \tableofcontents
    \clearpage



    %%%%%%%%%%%%%%%%%%%%%%%%%%%%%%%% Úvod %%%%%%%%%%%%%%%%%%%%%%%%%%%%%%%%
    \pagenumbering{arabic}
    \setcounter{page}{1}

    \section{Úvod}





    %%%%%%%%%%%%%%%%%%%%%%%%%%%%%%%% Návrh a implementace %%%%%%%%%%%%%%%%%%%%%%%%%%%%%%%%
    \section{Návrh a~implementácia}

    \subsection{Lexikálna analýza}
    Scanner slúži pre lexikálnu analýzu. Je implementovaný ako deterministický konečný automat, ktorý rozpoznáva všetky
    prichádzajúce tokeny. Uchováva informácie o tom či se jedná o komentár, identifikátor, textové bloky, relačné
    operátory alebo iné validné, poprípade nevalidné tokeny u ktorých nastane lexikálna chyba 1. V prípade validného
    tokenu na zaklade koncového stavu se vyplnia nasledujúce informácie v štruktúre token\_t:
    \begin{itemize}
        \item type  --  typ načítaného tokenu
        \item keyword -- keď typ je T\_KW, do premennej keyword sa uloží odpovedajúca hodnota
        \item attr -- hodnota tokenu
        \item attr.num\_i -- integer number
        \item attr.num\_f -- double number
        \item attr.id -- ostatné tokeny
    \end{itemize}
    V prípade že sa narazí na komentár je celý textový blok, ktorý je podľa správnej
    lexikálnej štruktúry chápaný ako komentár zahodený.


    \subsection{Syntaktická analýza}
    Parser je hlavným modulom prekladača, protože komunikuje se všetkými ostatnými modulami
    a riadi celú funkčnosť prekladača. Syntaktická analýza sa vykonáva zhora dolu metódou
    rekurzivného zostupu.
    Syntaktická analýza dostava od lexikálneho analyzátoru postupne tokeny, ktoré následne musia
    mať presnú syntaktickú štruktúru a postupnosť podla pravidiel LL-gramatiky.

    \subsection{Zpracovanie výrazov pomocou precedenčnej syntaktickej analýzy}
    Precedenčná syntaktická analýza je modul ktorý zaisťuje spracovanie výrazov metódou zdola hore.
    Vo svojom rozhraní obsahuje expression(), ktorú volá parser, keď chce precedenčnej analýze predať
    riadenie vo chvíli, kedy očakáva výraz.
    \subsubsection{Implementácia precedenčnej tabuľky}
    Postupne spracováva precedenčná analýza tokeny a pomocou precedenčnej tabuľky symbolov
    určuje precedenciu. Na základe tejto precedencie môžu nastať 5 stavov.
    1. Pri precedencií < pridávame na stack načítaný token spolu so znakom precedencie.
    2. Pri precedencií > redukujeme 2 výrazy na jeden a ukladáme ich typ podľa pravidla.
    3. Pri precedencií = zapíšeme načítaný znak z tokenu na stack.
    4. Pri precedencií c sme narazili na nesprávne poradie znakov a nastáva sémantická chyba.
    5. Pri precedencií s sme narazili na dva identifikátory,
                         redukujeme zvyšok výrazu a vraciame parseru riadenie a výsledný typ.

    \subsection{Sémantická analýza}


    \begin{itemize}
        \item Sémantické chyby pre nekompatibilitu typu priradenia, návratových hodnôt a predaných
        argumentov do funkcií se detekujú pomocou dvoch polí (ľavá strana typu a pravá strana typu). Po spracovaní
        určitého pravidla se vykoná porovnanie podľa sémantiky jazyka IFJ21.

        \item Kompatibilita typu vo výraze sa detekuje tým istým spôsobom, ale je použitý dve premenné typu char.

        \item Sémantické chyby pre nedefiníciu, redefiníciu sa detekujú pomocou tabuliek symbolov. Globálna tabulka
        symbolov je určená pre názvy funkcií. Pre názvy premenných je vytvorený zásobník
        tabuliek symbolov. Dôvodom implementácie zásobníka je riešenie problému s totožnými názvami premenných v
        rôznych rámcoch.
    \end{itemize}











    \subsection{Generovanie cielového kódu}
    Generovanie cieľového kódu IFJcode21 je implementovaný ako samostatný modul, ktorý je riadený syntaxou.
    Komponenty modulu su volané v parseri na základe pravidiel LL-gramatiky. Cieľový kód sa generuje
    priamo bez tvorby trojadresného kódu, ktorý sme nevytvárali na základe neoptimalizácie cieľového kódu.

    \subsubsection{Implementácia výpisu cieľového kódu}
    Na zaistenie výpisu cieľového kódu len za podmienky bezchybnej analýzy zapisujeme cieľový kód do dvoch textových
    blokov -- definície funkcií a volanie funkcií. Tieto dva textové bloky po úspešnej analýze\\ skonkatenujeme a
    vypíšeme na štandardný výstup.

    \subsubsection{Generovanie -- Deklarácie premenných}
    Deklarácie premenných a ich možný konflikt názvov (na základe výskytu toho istého názvu v rôznych rámcoch)
    sme implementovali vďaka obojsmernému radu v ktorom sa ukladá adresa elementu tabulky
    symbolov s príslušným identifikátorom. Element tabulky obsahuje unikátne číslo premennej, ktorý zaisťuje jedinečnosť
    názvu premennej.

    \subsubsection{Generovanie -- funkcie}
    Volanie funkcií je zaistené vygenerovaním kódu, ktorý predá funkcii argumenty pomocou dočasného rámcu, následne
    je vygenerovaný kód pre zavolanie funkcie. Pre správnu funkčnosť volanej funkcie je ihneď na začiatku generovaný kód,
    ktorý z dočasného rámca vytvôri lokálny rámec funkcie a pre všetky argumenty ktoré boli funkcii predané vytvori
    promenné s názvami podľa parametrov funkcie. Následne se generuje kód tela funkcie.

    \subsubsection{Generovanie -- výrazy}
    Generovanie kódu pre výraz sa začne vykonávať ihned po jeho redukcii. V priebehu redukcii je výraz zapísaný do obojstranného radu
    v postfixovom formáte. Jednotlivé elementy v rade nesú všetky potrebné
    informácie na generovanie výrazu - typ operátora, názov premennej či hodnotu konštanty. Generátor generuje len inštrukcie
    kódu Ifjcode21 ktoré využívajú zásobník. To znamená že hodnoty, medzivýsledky a následne výsledok výrazu sú
    uložené na zásobník.

    \subsubsection{Generovanie -- podmienky a cykly}
    Pre generovanie podmienok a cyklov využívame náveštia, ktoré sú taktiež reprezentované unikátnym číslom a
    názvom funkcie kde sa nachádzajú. Na zabránenie redeklarácie premenných sa telo cyklu zapisuje do dvoch rôznych textových
    blokov. Vyskytnuté deklarácie zapisujeme naďalej do bloku definícií funkcií. No zvyšný kód tela cyklu zapisujeme do
    tretieho pomocného textového bloku. Následne po vygenerovaní celého cyklu tieto dva bloky skonkatenujeme.

    \subsection{Prekladový systém}
    \subsubsection{CMake}
    CMake je multiplatformný nástroj na preklad zdrojových kódov. Nástroj sme vybrali na základe preferencií všetkých
    členov tímu. CMake nám predovšetkým pomáhal kompilovať a testovať výsledný program. Pravidlá pre preklad sú napísané v súbore
    CMakeLists.txt a po spustení nastroja CMake je automaticky vygenerovaný súbor Makefile.
    Na testovanie sme používali google testy, ktoré sme prekladali výhradne pomocou CMaku.

    \subsubsection{Skripty}
    Pre účely testovania boli vytvorené shellovské skripty. Jeden rozsiahly script, ktorý uľahčoval a automatizoval
    testovanie všetkých častí projektu. Taktiež sme vytvorili script pre preklad projektu a čistenie prebytočných súborov
    vzniknutých v dôsledku kompilácie projektu či testovacích suborov.

    \subsubsection{GNU Make}
    Zo zadania bolo požadované aby odovzdaný projekt obsahoval Makefile, ktorý s príkazom
    “make” preloží zdrojové súbory projektu a s “make clean” zmazal prebytočné súbory vzniknuté v dôsledku kompilácie.
    Tento nástroj nám taktiež pomáhal zabaliť celý projekt do jedného archívu zip.





    %%%%%%%%%%%%%%%%%%%%%%%%%%%%%%%% Speciální algoritmy a datové struktury %%%%%%%%%%%%%%%%%%%%%%%%%%%%%%
    \section{Špeciálne algoritmy a~dátové štruktúry}

    \subsection{Tabuľka s~rozptýlenými položkami}
    Túto dátovú štruktúru sme si zvolili vďaka časovej zložitosti
    (je reprezentovaná medzi O(1) až O(n)) a so skúsenosti štruktúry z
    predmetov IAL a IJC. Veľkosť tabuľky jsme zvolili 101.
    Ako unikátny kľúč pre prístup k dátam v tabulke slúži názov identifikátoru a názvy funkcií.
    Každý záznam v tabuľke obsahuje informácie o identifikátore. U premenných je uchovávaná aj informácia
    o hĺbke (redefinícia v zanorenejšom rámci kódu). Modul je implementovaný v súboroch
    symtable.h a symtable.c.


    \subsection{Pole rozptýlených tabuliek}
    TODO
    Modul je implementovaný v súboroch symstack.c a symstack.h

    \subsection{Obojsmerný rad}
    Obojsmerný rad je kombinácia zásobníka a radu. Je možné do neho vkladať aj odoberať dáta z oboch strán.
    Implementovali sme ho ako samostatný modul pre viac častí projektu.
    Je využívaný v generovaní kódu, kde slúži na uchovávanie postfixového výrazu či
    identifikátorov. Taktiež obsahuje informáciách o parametroch, argumentoch a navratovych hodnatách funkcií.
    Modul je implementovaný v súboroch queue.c a queue.h.


    \subsection{Dynamický reťazec}
    Pre uchovanie vygenerovaného kódu počas prekladu a prácu s identifikátormi sme vytvorili štruktúru string\_t.
    Pre obsluhu štruktúry sme vytvorili pomocné funkcie ako alokácia/dealokácia štruktúry,
    odstraňovanie, pridávanie a konkatenacia reťazcov.
    Modul je implementovaný v súboroch str.c a str.h.


    %%%%%%%%%%%%%%%%%%%%%%%%%%%%%%%% Práce v týmu %%%%%%%%%%%%%%%%%%%%%%%%%%%%%%%%
    \section{Práca v~týmu}
    Ihneď pri skladaní tímu sme všetci rozumeli že sa
    očakáva pravidelná a skorá práca na projekte čo sa nám nakoniec aj podarilo.
    Každý na projekte pracoval vždy s predstihom a darilo sa nam dodržiavať deadliny,
    ktoré sme si stanovili.


    \subsubsection{Komunikácia a spôsob práce v tíme}
    Pre komunikáciu sme používali výhradne komunikačnú platformu Discord, ktorý funguje na
    rovnakom princípe ako platforma Slack, ktorá sa používa profesionálne účely.
    Na danej platforme boli vytvorené komunikačné vlákna
    v ktorých boli založené "TODO” či error listy. Taktiež tam prebiehala bežná komunikácia či hlasové rozhovory s
    možnosťou zdieľania obrazovky vďaka čomu sme mohli vyriešiť mnoho problémov
    digitálne a tým aj veľmi rýchlo. Avšak aj napriek
    dobrej digitálnej komunikácii sme sa snažili mať čo najviac
    osobných stretnutí.

    \subsubsection{Verzovací systém a vývojové prostredie}
    Ako vývojové prostredie sme využili Clion a Vim. Vývoj prebiehal na platformách MacOs, Linux, Windows, no
    testovanie prebiehalo len na operačnom systému Linux. Ako verzovací systém sme použili git spolu s portálom
    GitHub.

    \subsection{Rozdelenie práce medzi členmi tímu}

    \textbf{Andrei:}
    \begin{itemize}
        \item  Lexikálna, sémantická a obecná syntaktická analýza
        \item  Organizácia a kontrola práce nad projektom
        \item  Tabulka symbolov
    \end{itemize}


    \textbf{Richard:}
    \begin{itemize}
        \item  Syntaktická a sémantická analýza pre výrazy
        \item  Precedenčná tabulka
        \item  Prezentácia
    \end{itemize}

    \textbf{Zdenek a Andrej:}
    \begin{itemize}
        \item  Generovanie kódu
        \item  Automatizácia testovania
        \item  Google testy, tvorba testov
        \item  Dokumentácia
    \end{itemize}


    %%%%%%%%%%%%%%%%%%%%%%%%%%%%%%%% Závěr %%%%%%%%%%%%%%%%%%%%%%%%%%%%%%%%
    \section{Záver}




    %%%%%%%%%%%%%%%%%%%%%%%%%%%%%%%% Citace %%%%%%%%%%%%%%%%%%%%%%%%%%%%%%%%
    \clearpage
    \bibliographystyle{czechiso}
    \renewcommand{\refname}{Literatura}
    \bibliography{dokumentace}



    %%%%%%%%%%%%%%%%%%%%%%%%%%%%%%%% Přílohy %%%%%%%%%%%%%%%%%%%%%%%%%%%%%%%%
    \clearpage


    %%%%%%%%%%%%%%%%%%%%%%%%%%%%%%%% Diagram konečného automatu %%%%%%%%%%%%%%%%%%%%%%%%%%%%%%%%

    \section*{Diagram konečného automatu specifikujúceho lexikálny analyzátor}
    \begin{figure}[!ht]
        \centering
        \vspace{-1.2cm}
        \includegraphics[width=0.95\linewidth]{src/FSM_PDF.pdf}
        \caption{Diagram konečného automatu specifikující lexikální analyzátor}
        \label{figure:fa_graph}
    \end{figure}

    %%%%%%%%%%%%%%%%%%%%%%%%%%%%%%%% LL -- gramatika %%%%%%%%%%%%%%%%%%%%%%%%%%%%%%%%
    \section*{LL -- gramatika}
    \begin{enumerate}[label=\textcolor{red}{\arabic*.}]
        \item \nonterm{prolog} \arrow{} \term{require} \term{t\unsc{}string} \nonterm{prog}

        \item \nonterm{prog} \arrow{} \term{global} \term{id} \term{:} \term{function} \term{(} \nonterm{arg\unsc{}T} \term{)} \nonterm{ret\unsc{}T} \nonterm{prog}

        \item \nonterm{prog} \arrow{} \term{function} \term{id} \term{(} \nonterm{arg} \term{)} \nonterm{ret\unsc{}T} \nonterm{stmt} \term{end} \nonterm{prog}

        \item \nonterm{prog} \arrow{} \term{id} \term{(} \nonterm{param} \term{)} \nonterm{prog}
        \item \nonterm{prog} \arrow{} \term{EOF}

        \item \nonterm{arg\unsc{}T} \arrow{} \nonterm{type} \nonterm{next\unsc{}arg\unsc{}T}
        \item \nonterm{arg\unsc{}T} \arrow{} \term{$\varepsilon$}

        \item \nonterm{next\unsc{}arg\unsc{}T} \arrow{} \term{,} \nonterm{type} \nonterm{next\unsc{}arg\unsc{}T}

        \item \nonterm{next\unsc{}arg\unsc{}T} \arrow{} \term{$\varepsilon$}

        \item \nonterm{ret\unsc{}T} \arrow{} \term{:} \nonterm{type} \nonterm{next\unsc{}ret\unsc{}T}

        \item \nonterm{ret\unsc{}T} \arrow{} \term{$\varepsilon$}

        \item \nonterm{next\unsc{}ret\unsc{}T} \arrow{} \term{,} \nonterm{type} \nonterm{next\unsc{}ret\unsc{}T}

        \item \nonterm{next\unsc{}ret\unsc{}T} \arrow{} \term{$\varepsilon$}

        \item \nonterm{arg} \arrow{} \term{id} \term{:} \nonterm{type} \nonterm{next\unsc{}arg}
        \item \nonterm{arg} \arrow{} \term{$\varepsilon$}

        \item \nonterm{next\unsc{}arg} \arrow{} \term{,} \term{id} \term{:} \nonterm{type} \nonterm{next\unsc{}arg}

        \item \nonterm{next\unsc{}arg} \arrow{} \term{$\varepsilon$}

        \item \nonterm{type} \arrow{} \term{integer}
        \item \nonterm{type} \arrow{} \term{number}
        \item \nonterm{type} \arrow{} \term{string}
        \item \nonterm{type} \arrow{} \term{nil}

        \item \nonterm{stmt} \arrow{} \term{if} \nonterm{expr} \term{then} \nonterm{stmt} \term{else} \nonterm{stmt} \term{end} \nonterm{stmt}

        \item \nonterm{stmt} \arrow{} \term{while} \nonterm{expr} \term{do} \nonterm{stmt} \term{end} \nonterm{stmt}

        \item \nonterm{stmt} \arrow{} \term{local} \term{id} \term{:} \nonterm{type} \nonterm{def\unsc{}var} \nonterm{stmt}

        \item \nonterm{stmt} \arrow{} \term{return} \nonterm{expr} \nonterm{next\unsc{}expr} \nonterm{stmt}

        \item \nonterm{stmt} \arrow{} \term{id} \nonterm{fork\unsc{}id} \nonterm{stmt}

        \item \nonterm{stmt} \arrow{} \term{$\varepsilon$}

        \item \nonterm{def\unsc{}var} \arrow{} \term{=} \nonterm{one\unsc{}assign}
        \item \nonterm{def\unsc{}var} \arrow{} \term{$\varepsilon$}

        \item \nonterm{one\unsc{}assign} \arrow{} \term{id} \term{(} \nonterm{param} \term{)}
        \item \nonterm{one\unsc{}assign} \arrow{} \nonterm{expr}

        \item \nonterm{param} \arrow{} \nonterm{param\unsc{}val} \nonterm{next\unsc{}param}
        \item \nonterm{param} \arrow{} \term{$\varepsilon$}

        \item \nonterm{param\unsc{}val} \arrow{} \term{id}
        \item \nonterm{param\unsc{}val} \arrow{} \nonterm{term}

        \item \nonterm{term} \arrow{} \term{t\unsc{}string}
        \item \nonterm{term} \arrow{} \term{t\unsc{}integer}
        \item \nonterm{term} \arrow{} \term{t\unsc{}number}
        \item \nonterm{term} \arrow{} \term{nil}

        \item \nonterm{next\unsc{}param} \arrow{} \term{,} \nonterm{param\unsc{}val} \nonterm{next\unsc{}param}

        \item \nonterm{next\unsc{}param} \arrow{} \term{$\varepsilon$}

        \item \nonterm{next\unsc{}expr} \arrow{} \term{,} \nonterm{expr} \nonterm{next\unsc{}expr}
        \item \nonterm{next\unsc{}expr} \arrow{} \term{$\varepsilon$}

        \item \nonterm{fork\unsc{}id} \arrow{} \term{(} \nonterm{param} \term{)}
        \item \nonterm{fork\unsc{}id} \arrow{} \nonterm{next\unsc{}id}

        \item \nonterm{next\unsc{}id} \arrow{} \term{,} \term{id} \nonterm{next\unsc{}id}
        \item \nonterm{next\unsc{}id} \arrow{} \term{=} \nonterm{mult\unsc{}assign}

        \item \nonterm{mult\unsc{}assign} \arrow{} \term{id} \term{(} \nonterm{param} \term{)}
        \item \nonterm{mult\unsc{}assign} \arrow{} \nonterm{expr} \nonterm{next\unsc{}expr}
    \end{enumerate}

    %%%%%%%%%%%%%%%%%%%%%%%%%%%%%%%% LL -- tabulka %%%%%%%%%%%%%%%%%%%%%%%%%%%%%%%%
    \newpage
    \section*{LL -- tabulka}

    \begin{table}[htb]
        \begin{adjustwidth}{-0.2cm}{}
            \begin{tabular}{!{\vrule width 2pt}>{\columncolor{LightCyan}}P{2.3cm}!{\vrule width 2pt}P{2.5mm}|P{2.5mm}|P{2.5mm}|P{2.5mm}|P{2.5mm}|P{2.5mm}|P{2.5mm}|P{2.5mm}|P{2.5mm}|P{2.5mm}|P{2.5mm}|P{2.5mm}|P{2.5mm}|P{2.5mm}|P{2.5mm}|P{2.5mm}|P{2.5mm}|P{2.5mm}|P{2.5mm}|P{2.5mm}|P{2.5mm}!{\vrule width 2pt}}
                \Xhline{5\arrayrulewidth} \rowcolor{LightCyan}
                \cellcolor{White}& \rot{require} & \rot{global} & \rot{function} & \rot{id} & \rot{integer}
                & \rot{string} & \rot{number} & \rot{nil} & \rot{t\unsc{}integer} & \rot{t\unsc{number}} & \rot{t\unsc{}string} & \rot{if} & \rot{while} & \rot{local} & \rot{return}  & \rot{:} & \rot{=} & \rot{(} & \rot{,} & \rot{EOF} & \rot{\$}\\
                \Xhline{5\arrayrulewidth}
                \nonterm{prolog} & 1 & & & & & & & & & & & & & & & & & & & & \\
                \hline
                \nonterm{prog} & & 2 & 3 & 4 & & & & & & & & & & & & & & & & 5 &\\
                \hline
                \nonterm{arg\unsc{}T} & & & & & 6 & 6 & 6 & 6 & & & & & & & & & & & & & 7 \\
                \hline
                \nonterm{next\unsc{}arg\unsc{}T} & & & & & & & & & & & & & & & & & & & 8 & & 9 \\
                \hline
                \nonterm{ret\unsc{}T} & & & & & & & & & & & & & & & & 10 & & & & & 11 \\
                \hline
                \nonterm{next\unsc{}ret\unsc{}T} & & & & & & & & & & & & & & & & & & & 12 & & 13 \\
                \hline
                \nonterm{arg} & & & & 14 & & & & & & & & & & & & & & & & & 15 \\
                \hline
                \nonterm{next\unsc{}arg} & & & & & & & & & & & & & & & & & & & 16 & & 17 \\
                \hline
                \nonterm{type} & & & & & 18 & 20 & 19 & 21 & & & & & & & & & & & & &\\
                \hline
                \nonterm{stmt} & & & & 26 & & & & & & & & 22 & 23 & 24 & 25 & & & & & & 27 \\
                \hline
                \nonterm{def\unsc{}var} & & & & & & & & & & & & & & & & & 28 & & & & 29 \\
                \hline
                \nonterm{one\unsc{}assign} & & & & 30 & & & & & & & & & & & & & & & & & 31 \\
                \hline
                \nonterm{param} & & & & 32 & & & & 32 & 32 & 32 & 32 & & & & & & & & & & 33 \\
                \hline
                \nonterm{param\unsc{}val} & & & & 34 & & & & 35 & 35 & 35 & 35 & & & & & & & & & &\\
                \hline
                \nonterm{term} & & & & & & & & 39 & 37 & 38 & 36 & & & & & & & & & &\\
                \hline
                \nonterm{next\unsc{}param} & & & & & & & & & & & & & & & & & & & 40 & & 41 \\
                \hline
                \nonterm{next\unsc{}expr} & & & & & & & & & & & & & & & & & & & 42 & & 43\\
                \hline
                \nonterm{fork\unsc{}id} & & & & & & & & & & & & & & & & & 45 & 44 & 45 & & \\
                \hline
                \nonterm{next\unsc{}id} & & & & & & & & & & & & & & & & & 47 & & 46 & & \\
                \hline
                \nonterm{mult\unsc{}assign} & & & & 48 & & & & & & & & & & & & & & & & & 49 \\
                \hline
                \Xhline{5\arrayrulewidth}
            \end{tabular}
        \end{adjustwidth}
    \end{table}


    %%%%%%%%%%%%%%%%%%%%%%%%%%%%%%%% PRECEDENCNI TABULKA %%%%%%%%%%%%%%%%%%%%%%%%%%%%%%%%
    \newpage
    \section*{Precedenčná tabuľka}
    \newcommand{\bE}{\textbf{E}}
    \begin{table}[!htbp]
        \centering
        \begin{tabular}{l@{\hskip 1in} l@{\hskip 1in} l}
            $\color{red} 1.$ \bE{} \space\arrow{} i               & $\color{red} 6.$  \bE{} \space\arrow{} \bE{} $*$ \bE{}  & $\color{red} 11.$ \bE{} \space\arrow{} \bE{} $<$ \bE{}     \\
            $\color{red} 2.$ \bE{} \space\arrow{} ( \bE{} )         & $\color{red} 7.$  \bE{} \space\arrow{} \bE{} $/$ \bE{}  & $\color{red} 12.$ \bE{} \space\arrow{} \bE{} $>=$ \bE{}    \\
            $\color{red} 3.$ \bE{} \space\arrow{} \# \bE{}         & $\color{red} 8.$  \bE{} \space\arrow{} \bE{} $//$ \bE{} & $\color{red} 13.$ \bE{} \space\arrow{} \bE{} $<=$ \bE{}    \\
            $\color{red} 4.$ \bE{} \space\arrow{} \bE{} $+$ \bE{} & $\color{red} 9.$  \bE{} \space\arrow{} \bE{} .. \bE{}   & $\color{red} 14.$ \bE{} \space\arrow{} \bE{} $==$ \bE{}    \\
            $\color{red} 5.$ \bE{} \space\arrow{} \bE{} $-$ \bE{} & $\color{red} 10.$ \bE{} \space\arrow{} \bE{} $>$ \bE{}  & $\color{red} 15.$ \bE{} \space\arrow{} \bE{} $\sim=$ \bE{} \\
        \end{tabular}
    \end{table}

    \begin{center}
        \begin{adjustwidth}{-0.4cm}{}
            \begin{tabular}{!{\vrule width 2pt} >{\columncolor{LightCyan}}P{5.5mm} !{\vrule width 2pt} P{5.5mm} | P{5.5mm} | P{5.5mm} | P{5.5mm} | P{5.5mm} | P{5.5mm} | P{5.5mm} | P{5.5mm} | P{5.5mm} | P{5.5mm} | P{5.5mm} | P{5.5mm} | P{5.5mm} | P{5.5mm} | P{5.5mm} | P{5.5mm} | P{5.5mm} !{\vrule width 2pt}}
                \Xhline{5\arrayrulewidth}\rowcolor{LightCyan}
                \cellcolor{White}& \textbf{\#} & \textbf{*} & \textbf{/} & \textbf{//} & \textbf{+} & \textbf{-} & \textbf{..} & $\bm{<}$ & $\bm{<=}$ & $\bm{>}$ & $\bm{>=}$ & $\bm{==}$ & $\bm{\sim=}$ & \textbf{(} & \textbf{)} & \textbf{i} & \textbf{\$} \\ [0.6ex]
                \Xhline{5\arrayrulewidth}
                \textbf{\#}      & $<$ & $>$ & $>$ & $>$ & $>$ & $>$ & $>$ & $>$ & $>$ & $>$ & $>$ & $>$ & $>$ & $<$ & $>$ & $<$ & $>$ \\ [0.5ex]
                \hline
                \textbf{*}       & $<$ & $>$ & $>$ & $>$ & $>$ & $>$ & $>$ & $>$ & $>$ & $>$ & $>$ & $>$ & $>$ & $<$ & $>$ & $<$ & $>$ \\ [0.5ex]
                \hline
                \textbf{/}       & $<$ & $>$ & $>$ & $>$ & $>$ & $>$ & $>$ & $>$ & $>$ & $>$ & $>$ & $>$ & $>$ & $<$ & $>$ & $<$ & $>$ \\ [0.5ex]
                \hline
                \textbf{//}      & $<$ & $>$ & $>$ & $>$ & $>$ & $>$ & $>$ & $>$ & $>$ & $>$ & $>$ & $>$ & $>$ & $<$ & $>$ & $<$ & $>$ \\ [0.5ex]
                \hline
                \textbf{+}       & $<$ & $<$ & $<$ & $<$ & $>$ & $>$ & $>$ & $>$ & $>$ & $>$ & $>$ & $>$ & $>$ & $<$ & $>$ & $<$ & $>$ \\ [0.5ex]
                \hline
                \textbf{-}       & $<$ & $<$ & $<$ & $<$ & $>$ & $>$ & $>$ & $>$ & $>$ & $>$ & $>$ & $>$ & $>$ & $<$ & $>$ & $<$ & $>$ \\ [0.5ex]
                \hline
                \textbf{..}      & $<$ & $<$ & $<$ & $<$ & $<$ & $<$ & $<$ & $>$ & $>$ & $>$ & $>$ & $>$ & $>$ & $<$ & $>$ & $<$ & $>$ \\ [0.5ex]
                \hline
                $\bm{<}$     & $<$ & $<$ & $<$ & $<$ & $<$ & $<$ & $<$ & $>$ & $>$ & $>$ & $>$ & $>$ & $>$ & $<$ & $>$ & $<$ & $>$ \\ [0.5ex]
                \hline
                $\bm{<=}$    & $<$ & $<$ & $<$ & $<$ & $<$ & $<$ & $<$ & $>$ & $>$ & $>$ & $>$ & $>$ & $>$ & $<$ & $>$ & $<$ & $>$ \\ [0.5ex]
                \hline
                $\bm{>}$     & $<$ & $<$ & $<$ & $<$ & $<$ & $<$ & $<$ & $>$ & $>$ & $>$ & $>$ & $>$ & $>$ & $<$ & $>$ & $<$ & $>$ \\ [0.5ex]
                \hline
                $\bm{>=}$    & $<$ & $<$ & $<$ & $<$ & $<$ & $<$ & $<$ & $>$ & $>$ & $>$ & $>$ & $>$ & $>$ & $<$ & $>$ & $<$ & $>$ \\ [0.5ex]
                \hline
                $\bm{==}$    & $<$ & $<$ & $<$ & $<$ & $<$ & $<$ & $<$ & $>$ & $>$ & $>$ & $>$ & $>$ & $>$ & $<$ & $>$ & $<$ & $>$ \\ [0.5ex]
                \hline
                $\bm{\sim=}$ & $<$ & $<$ & $<$ & $<$ & $<$ & $<$ & $<$ & $>$ & $>$ & $>$ & $>$ & $>$ & $>$ & $<$ & $>$ & $<$ & $>$ \\ [0.5ex]
                \hline
                \textbf{(}       & $<$ & $<$ & $<$ & $<$ & $<$ & $<$ & $<$ & $<$ & $<$ & $<$ & $<$ & $<$ & $<$ & $<$ & $=$ & $<$ &  e  \\ [0.5ex]
                \hline
                \textbf{)}       & $>$ & $>$ & $>$ & $>$ & $>$ & $>$ & $>$ & $>$ & $>$ & $>$ & $>$ & $>$ & $>$ &  e  & $>$ &  s  & $>$ \\ [0.5ex]
                \hline
                \textbf{i}       &  e  & $>$ & $>$ & $>$ & $>$ & $>$ & $>$ & $>$ & $>$ & $>$ & $>$ & $>$ & $>$ &  e  & $>$ &  s  & $>$ \\ [0.5ex]
                \hline
                \textbf{\$}      & $<$ & $<$ & $<$ & $<$ & $<$ & $<$ & $<$ & $<$ & $<$ & $<$ & $<$ & $<$ & $<$ & $<$ &  e  & $<$ &  e  \\ [0.5ex]
                \Xhline{5\arrayrulewidth}
            \end{tabular}
        \end{adjustwidth}
    \end{center}

    LEGENDA: \\
    $<$  \space-\space insert to stack with shift \\
    $>$  \space-\space reduction \\
    =    \space-\space insert to stack \\
    e    \space-\space error \\
    s    \space-\space special case (end of expression)\\


\end{document}

